\chapter{Elastic Load Balancing (ELB)}\label{ch:elastic-load-balancing}

Elastic Load Balancing is used to automatically scale your EC2 instances to meet any changes in demand from your users.
ELB can be used to both scale up or down the instance's resources to ensure that your application is kept performant and
available regardless of how many users visit.
It is also possible to scale down the resources again, after the spike in usage has passed, to save money when it comes
to billing and ensure you aren't paying for any additional resources that are not being used.

1. Creation of new AMI

- Selection of AMI previously created and then "Create Instance from AMI" option (elb/elb-instance-from-ami)
- Given name of "Group4-EC2-Instance-2" (elb/elb-instance-2-name)
- AMI selected to use same instance as "Group4-EC2" so everything is the same (elb/elb-instance-2-name)
- Same keypair used, so it is easier to switch between both instances (elb/elb-instance-2-type-and-keypair)
    Network Settings
- VPC set to be Group4 VPC  (created in Section~\ref{ch:vpc}) (elb/elb-instance-2-network-settings)
- Subnet set to be "Public Subnet 2" (elb/elb-instance-2-network-settings)
- Public IP auto-assigned (elb/elb-instance-2-network-settings)
- Existing security group of "Group4-Security-Group" selected to allow HTTP, HTTPS, SSH and MySQL traffic on the
instance (elb/elb-instance-2-network-settings)
- Storage remains the same as previous instance (elb/elb-instance-2-storage-config)
- Second instance now created (elb/elb-instance-2-created)

2. Creation of target group

Now that there is 2 instances, they both need to be stored within a target group.

- The Target Group type of "Instances" is selected - this will store both running instances of Digital-Ink in a group
to be handled by the Load Balancer (elb/elb-target-group-basic-config)
- Target Group name of "Group4-Target-Group" given (elb/elb-target-group-basic-config)
- Protocol of HTTP and port 80 given (elb/elb-target-group-basic-config)
- Group4-VPC selected (elb/elb-vpc)
- The next screen asks for instances that will be registered to the Target Group to be selected
- The 2 running instances of Digital Ink are shown and added as targets through the "Include as pending below"
button (elb/elb-register-targets)
- Target group now created, and contains both EC2 instances within them (elb/elb-target-group-created)

3. Creation of Load Balancer

- The Target groups are not associated with a Load Balancer, so one will be created as an Application Load Balancer
(elb/elb-load-balancer-created)
- Given the name of "Group4-Load-Balancer' (elb/elb-basic-config)
- Set to be internet facing and will handle ipv4 addresses (elb/elb-network-mapping)
- The VPC is selected and public subnet 1 and 2 are applied within each, which are in us-east-1a and us-east-1b
respectively (Say this makes the web app highly available) (elb/elb-vpc)
- The security group of Group4-Security-Group is selected, which allows HTTP, HTTPS, SSH and MySQL traffic within
the load balancer (elb/elb-security-groups-and-listeners)
- In the event that either of the web apps goes down, it will be forwarded to the Group4-Target-Group, which will then
display the available instance (elb/elb-security-groups-and-listeners)
- Could not create Global Accelerator due to permissions issue (Future improvement) (elb/elb-accelerator)
- Summary (elb/elb-summary)
- Load balancer created (elb/elb-created)
- When the load balancer is visited at
\href{http://http://group4-load-balancer-1914525647.us-east-1.elb.amazonaws.com/}{http://http://group4-load-balancer-1914525647.us-east-1.elb.amazonaws.com/},
the website is shown (elb/elb-working)














%\begin{figure}[!htbp]
%    \centering
%    \includegraphics[width=\textwidth]{resources/find-image}
%    \caption{Selection of EC2 OS Image.}
%    \label{fig:ec2-os}
%\end{figure}