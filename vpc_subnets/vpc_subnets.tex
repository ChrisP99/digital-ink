\chapter{Virtual Private Cloud (VPC)}\label{ch:vpc}

Amazon VPC allows for AWS resources to be launched in a virtual network that has been custom defined and configured.
This virtual network is similar to a traditional network which operates within your own physical data center, with the
added benefits of the scalable AWS infrastructure~\parencite{amazon2022what}.
A VPC can have multiple assigned subnets, which are a range of IP addresses accessible in the VPC\@.

The first step of migrating the web app to the cloud was creating a VPC to deploy and manage AWS resources for the app.
To do this, the VPC wizard must be used.
This is accessed by navigating through the VPC Management Console to the "Your VPCs" page, and then clicking the Create
VPC button, which can be seen in Figure~\ref{fig:vpc-wizard}.
The AWS Academy sandbox already provides a default VPC, however, a new VPC needs to be created with the necessary
configurations.

\begin{figure}[!htbp]
    \centering
    \includegraphics[width=\textwidth]{resources/vpc/your-vpcs-before}
    \caption{"Your VPCs" page.}
    \label{fig:vpc-wizard}
\end{figure}

The wizard presents us with step one of creating a VPC: selecting a VPC configuration.
There are several options for this, where each configuration has the following features:

\begin{itemize}
    \item VPC with a Single Public Subnet
    \item VPC with Public and Private Subnets
    \item VPC with Public and Private Subnets and Hardware VPN Access
    \item VPC with a Private Subnet Only and Hardware VPN Access
\end{itemize}
